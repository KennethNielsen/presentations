\documentclass{beamer}

%\usetheme{Warsaw}
\usecolortheme{orchid}

%\usepackage{listings}
%\lstset{language=Python}

\usepackage{color}
\definecolor{darkred}{RGB}{180,0,0}

\usepackage{minted}

\title{Python decorators}
\subtitle{Powerful and not all that magical after all}
\author{Kenneth Nielsen\inst{1}}
\institute
{
  \inst{1}%
  Center for Individual Nanoparticle Functionality (CINF)\\
  Institute of Physics\\
  Technical University of Denmark (DTU)
}
\date{
  Python and beers presentation,\\
  March 20, 2015}
\subject{Computer Science}

\begin{document}

\frame{\titlepage}

\begin{frame}
  \frametitle{About me}
  \begin{itemize}
    \item Name is Kenneth Nielsen and I work at CINF
    \item Free and open source software (FOSS) enthusiast
    \item Long time FOSS translator
    \item BIG TIME Pythonista
    \item PyG3T https://launchpad.net/pyg3t. Tools for the translation
      workflow.
    \item PoProofRead https://launchpad.net/poproofread. Translation
      proofreading tool
    \item SoCo https://github.com/SoCo/SoCo. Driver for Sonos speakers.
    \item PyExpLabSys https://github.com/CINF/PyExpLabSys.
  \end{itemize}
\end{frame}

\begin{frame}
  \frametitle{PyExpLabSys}
  {\color{blue}Py}thon {\color{blue}Exp}erimental {\color{blue}Lab}
  {\color{blue}Sys}tems (Best acronym ever!)
  \begin{itemize}
  \item Drivers for experimental equipment
  \item File parsers for 3-rd party data files
  \item Infrastructure for a network driven, decentralized, lab:
    \begin{itemize}
    \item Database logging
    \item Wrappers around sockets and socket servers
    \item Modules for live data transfer to webpages
    \end{itemize}
  \end{itemize}
\end{frame}

\begin{frame}[fragile]
  \frametitle{Decorators}
\begin{minted}{python}
@my_decorator
def my_function():
    pass
\end{minted}

\begin{exampleblock}{}
  \begin{itemize}
  \item Wraps (or decorates) Python functions
  \item Can be used to add functionality to functions
  \item Start-end, enter-exit, startup-teardown type code (or just one or the other)
  \item Bah! Lets talk about sandwiches
  \end{itemize}
\end{exampleblock}
\end{frame}


% Sandwiches
\begin{frame}[fragile]
  \frametitle{Sandwich machine}
\begin{minted}{python}
def white_bread_blt():
    print 'Apply bottom half of white bun'
    print 'Apply Bacon'
    print 'Apply Lettuce'
    print 'Apply Tomatoes'
    print 'Apply top half of white bun'

def full_wheat_blt():
    print 'Apply bottom half of full wheat bun'
    print 'Apply Bacon'
    print 'Apply Lettuce'
    print 'Apply Tomatoes'
    print 'Apply top half of full wheat bun'

...
\end{minted}
\pause
Kind of annoying, right? \pause Because it's dry!
\end{frame}

% DRY
\begin{frame}
  \Huge{}
  {\color{blue}D}\uncover<2>{on't}\\
  {\color{blue}R}\uncover<2>{epeat}\\
  {\color{blue}Y}\uncover<2>{ourself}\\
  \normalsize{}
\end{frame}

% Sandwiches again
\begin{frame}[fragile]
  \frametitle{Sandwich machine}
\begin{minted}{python}
def white_bread_blt():
    print 'Apply bottom half of white bun'
    print 'Apply Bacon'
    print 'Apply Lettuce'
    print 'Apply Tomatoes'
    print 'Apply top half of white bun'

def full_wheat_blt():
    print 'Apply bottom half of full wheat bun'
    print 'Apply Bacon'
    print 'Apply Lettuce'
    print 'Apply Tomatoes'
    print 'Apply top half of full wheat bun'

...
\end{minted}
\pause Aha! We can fix that with a sub function.
\end{frame}

% Sandwiches with sub function
\begin{frame}[fragile]
  \frametitle{Sandwich machine with sub function}
\begin{minted}{python}
def white_bread_blt():
    print 'Apply bottom half of white bun'
    blt()
    print 'Apply top half of white bun'

def full_wheat_blt():
    print 'Apply bottom half of full wheat bun'
    blt()
    print 'Apply top half of full wheat bun'

...

def blt():
    print 'Apply Bacon'
    print 'Apply Lettuce'
    print 'Apply Tomatoes'
\end{minted}
\pause
But what if it was the other way around?
\end{frame}

% Different kinds of sandwiches
\begin{frame}[fragile]
  \frametitle{Sandwich machine with sub function}
\begin{minted}{python}
def full_wheat_blt():
    print 'Apply bottom half of full wheat bun'
    print 'Apply Bacon'
    print 'Apply Lettuce'
    print 'Apply Tomatoes'
    print 'Apply top half of full wheat bun'

def full_wheat_ham_and_cheese():
    print 'Apply bottom half of full wheat bun'
    print 'Apply Ham'
    print 'Apply Cheese'
    print 'Apply top half of full wheat bun'

...
\end{minted}
\pause
Could try with sub functions
\end{frame}

% Different kinds of sandwiches with sub functions
\begin{frame}[fragile]
  \frametitle{Sandwich machine with sub function}
\begin{minted}{python}
def full_wheat_blt():
    full_wheat_bottom()
    print 'Apply Bacon'
    print 'Apply Lettuce'
    print 'Apply Tomatoes'
    full_wheat_top()

def full_wheat_ham_and_cheese():
    full_wheat_bottom()
    print 'Apply Ham'
    print 'Apply Cheese'
    full_wheat_top()

def full_wheat_bottom():
    print 'Apply bottom half of full wheat bun'
def full_wheat_top():
    print 'Apply top half of full wheat bun'
\end{minted}
\pause
Yuck!
\end{frame}

% The revelation
\begin{frame}
  \frametitle{Yay, decorators!}
  \begin{itemize}
  \item Turns out, this is one of the problems that can be solved more
    elegantly with decorators
  \item But before we dive in, lets talk a little about functions and objects
  \end{itemize}
\end{frame}

% World of objects
\begin{frame}[fragile]
  \frametitle{Objects can be passed through functions and assigned other names}
\begin{minted}{python}
def pass_through(obj):
    return obj

my_list = ['Love Python and beers']
my_int = 47
my_other_int = my_int

my_new_list = pass_through(my_list)
my_new_int = pass_through(my_int)

my_list is my_new_list
my_int is my_new_int
my_int is my_other_int
# All returns True
\end{minted}
\end{frame}

% Objects has methods and properties
\begin{frame}[fragile]
  \frametitle{Objects has methods and properties (even ints)}
\begin{minted}{python}
### Method on an int
my_int.bit_length
# <built-in method bit_length of int object at 0x155..>

my_int.bit_length()
# 6

### Property on an int
my_int.real
# 47

### Method on a list
my_list.append
# <built-in method append of list object at 0x7fe..>
\end{minted}
\end{frame}

% Functions are objects too
\begin{frame}[fragile]
  \frametitle{Functions are objects too, part 1}
\begin{minted}{python}
pass_through
# <function pass_through at 0x7f706ec4f758>

### With methods
pass_through.__format__
# <built-in method __format__ of function object at 0x7f..>
pass_through.__format__('')
# '<function pass_through at 0x7f706ec4f758>'

### And properties
pass_through.__name__
# 'pass_through'
\end{minted}
\end{frame}

% Functions are objects too
\begin{frame}[fragile]
  \frametitle{Functions are objects too, part 2}
\begin{minted}{python}
### and they can be assigned a different name
my_completely_unrelated_function = pass_through
print my_completely_unrelated_function(9)
# 9

### so it does the same, and has the same attributes
my_completely_unrelated_function.__name__
# 'pass_through'

### because it's the same object
\end{minted}
\end{frame}

% Functions are objects too
\begin{frame}[fragile]
  \frametitle{AND functions can be passed through functions}
\begin{minted}{python}
def my_new_func():
    print 'Python and beers'

what_a_ride = pass_through(my_new_func)
what_a_ride.__name__
# 'my_new_func'

what_a_ride()
# Python and beers
\end{minted}
\begin{block}{\vspace*{-3ex}}
  And now back to decorators!  
\end{block}
\end{frame}

% Decorators are just syntactic sugar
\begin{frame}[fragile]
  \frametitle{Decorators are just syntactic sugar}
\begin{minted}{python}
@my_decorator
def my_function():
    pass
\end{minted}
\begin{alertblock}{\vspace*{-3ex}}
  Is 100\% equivalent to
\end{alertblock}
\begin{minted}{python}
def my_function():
    pass
my_function = my_decorator(my_function)
\end{minted}
\end{frame}

% Down the rabbit hole
\begin{frame}[fragile]
  \frametitle{The essence of a decorator}
\begin{minted}{python}
@my_decorator
def my_function():
    pass
\end{minted}
\begin{block}{\vspace*{-3ex}}
\begin{itemize}
\item A decorator is the function (my\_decorator), the function it is applied to (my\_function), is passed through, right after it is defined
\end{itemize}
\end{block}
\begin{minted}{python}
def my_function():
    pass
my_function = my_decorator(my_function)
\end{minted}
\begin{block}{\vspace*{-3ex}}
  Enough talk, lets see one already!
\end{block}
\end{frame}

% My first decorator
\begin{frame}[fragile]
  \frametitle{A do nothing decorator (ex1.py)}
\begin{minted}{python}
def my_decorator(function):
    return function

@my_decorator
def my_function():
    print 'python and beers'

my_function()
# python and beers
\end{minted}
\end{frame}

% My first decorator
\begin{frame}[fragile]
  \frametitle{A needy decorator (ex2.py)}
\begin{minted}{python}
def my_decorator(function):
    print 'Look at me, look at me, I\'m a decorator'
    return function

@my_decorator
def my_function():
    print 'python and beers'

my_function()
# Look at me, look at me, I'm a decorator
# python and beers
\end{minted}
\begin{block}{\vspace*{-3ex}}
  And now, for some actually useful decorators
\end{block}
\end{frame}

% A timer
\begin{frame}[fragile]
  \frametitle{A timer (ex3.py)}
\begin{minted}{python}
import time

def time_me(function):
    def inner_function(n):
        t0 = time.time()
        out = function(n)
        print 'Duration', str(time.time() - t0), 's'
        return out
    return inner_function

@time_me
def square(n):
    return n ** 2

print square(10)
# Duration 1.28746032715e-05 s
# 100
\end{minted}
\end{frame}

% Should be flexible
\begin{frame}
  \frametitle{Decorators should be flexible}
  \begin{itemize}
  \item In order for decorators to be really useful, they should be written to
    be flexible
  \item Take any combination of arguments
  \item (Leave as little footprint on the function as possible) we won't go
    that much into that
  \end{itemize}
\end{frame}

% A flexible timer
\begin{frame}[fragile]
  \frametitle{A flexible timer (ex4.py)}
\begin{minted}{python}
import time

def time_me(function):
    def inner_function(*args, **kwargs):
        t0 = time.time()
        out = function(*args, **kwargs)
        print 'Duration', str(time.time() - t0), 's'
        return out
    return inner_function

@time_me
def square(n, repeat=1):
    for _ in range(repeat):
        out = n ** 2
    return out
\end{minted}
\end{frame}

% A flexible timer
\begin{frame}[fragile]
  \frametitle{A flexible timer (ex4.py)}
\begin{minted}{python}
print square(10)
print
print square(10, 10**6)

#Duration 1.4066696167e-05 s
#100
#
#Duration 0.0811638832092 s
#100
\end{minted}
\end{frame}

% A flexible timer with cumulative sum
\begin{frame}[fragile]
  \frametitle{A flexible timer with cumulative sum (ex5.py), using attributes
    on the function}
\begin{minted}{python}
import time

def time_me_cumulative(function):
    def inner_function(*args, **kwargs):
        t0 = time.time()
        out = function(*args, **kwargs)
        delta = time.time() - t0
        inner_function.cumsum += delta
        print "This run:", delta, "in total:",\
            inner_function.cumsum
        return out
    inner_function.cumsum = 0
    return inner_function
\end{minted}
\end{frame}

% A flexible timer with cumulative sum
\begin{frame}[fragile]
  \frametitle{A flexible timer with cumulative sum (ex5.py), application}
\begin{minted}{python}
@time_me_cumulative
def square(n, repeat=1):
    for _ in range(repeat):
        out = n ** 2
    return out

square(10, 10**6)
square(10, 10**6)
#This run: 0.0809688568115 in total: 0.0809688568115
#This run: 0.0796308517456 in total: 0.160599708557
\end{minted}
\end{frame}

\begin{frame}
  \frametitle{Don't re-implement common tools}
  \begin{itemize}
  \item At this point we are well on out way to implementing a full profiler
  \item If you are thinking, ``surely other will have needed this before me'',
    you are most likely right. For advanced functionality \textbf{use existing and commony used implementations}
  \item You are like you miss a few corner cases that will hurt you later
  \end{itemize}
  \begin{block}{}
    Have a look at $\texttt{cProfile}$ in the standard library or
    \texttt{line\_profiler} in pip
  \end{block}
\end{frame}

\begin{frame}
  \frametitle{Debugging by print statements}
  \begin{itemize}
  \item Often debugging is done by adding print statements at appropriate
    points
  \item Often on entry and exit from a function
  \item It can be cumbersome to add and remove all of these
  \item Sounds like a job for decorator
  \end{itemize}
\end{frame}

% A call spec decorator
\begin{frame}[fragile]
  \frametitle{A call specification decorator (ex6.py), the decorator}
\begin{minted}{python}
def callstring(*args, **kwargs):
    arglist = [str(arg) for arg in args]
    arglist += [key + "=" + str(kwarg)
                for key, kwarg in kwargs.items()]
    return ', '.join(arglist)

def spec_me(function):
    def inner_function(*args, **kwargs):
        argstring = callstring(*args, **kwargs)
        print '+{}({})'.format(function.__name__,
                               argstring)
        out = function(*args, **kwargs)
        print '>{}'.format(out)
        return out
    return inner_function
\end{minted}
\end{frame}

% A call spec decorator
\begin{frame}[fragile]
  \frametitle{A call specification decorator (ex6.py), application}
\begin{minted}{python}
@spec_me
def square(n, repeat=1):
    for _ in range(repeat):
        out = n ** 2
    return out

square(10)
#+square(10)
#>100

square(10, repeat=2)
#+square(10, repeat=2)
#>100
\end{minted}
\end{frame}

% A call spec decorator with levels
\begin{frame}[fragile]
  \frametitle{A call spec decorator with levels (ex7.py)\\(Trick with global variable, left as an exercise)}
\begin{minted}{python}

@spec_me
def square(n):
    return n ** 2

@spec_me
def speak_squares(n):
    return '{} squared is {}'.format(n, square(n))

speak_squares(10)

#+speak_squares(10)
#|+square(10)
#|>100
#>10 squared is 100
\end{minted}
\end{frame}

\begin{frame}
  \frametitle{A function with memory}
  \begin{itemize}
  \item Wait a minute!
  \item The ability to look at input and output and to save values on the
    functions
  \item That sounds like the foundation for a function with memory
  \end{itemize}
\end{frame}

% A call spec decorator with levels
\begin{frame}[fragile]
  \frametitle{A memory decorator (ex8.py)}
\begin{minted}{python}
def memory(function):
    cache = {}
    def inner_function(n):
        if n not in cache:
            cache[n] = function(n)
        return cache[n]
    return inner_function

### Same result as

def memory2(function):
    def inner_function(n):
        if n not in inner_function.cache:
            inner_function.cache[n] = function(n)
        return inner_function.cache[n]
    inner_function.cache = {}
    return inner_function
\end{minted}
\end{frame}

% A call spec decorator with levels in action
\begin{frame}[fragile]
  \frametitle{A memory decorator (ex8.py) in action}
\begin{minted}{python}
from ex3 import time_me

@time_me
@memory
def expensive(n):
    for n in range(10**7):
        out = n ** 2
    return out

out1 = expensive(8)
out2 = expensive(8)
print out1 == out2

# Duration 0.715964078903 s
# Duration 9.53674316406e-07 s
# True
\end{minted}
\end{frame}

\begin{frame}
  \frametitle{Neat, but its not that simple}
  \begin{itemize}
  \item While this works nicely for certain kinds of functions, that are a lot
    of corner cases and gochas
  \item What if the input was a list, you can't use a mutable object as a
    dictionary key in $\texttt{cache}$. (Make a copy as a immutable type)
  \item What if the input was a big numpy array, do we really want to keep a
    copy of that in memory? (hashing)
  \item What if you wanted your function to remember across script runs? How
    do you save input and output?
  \end{itemize}
\end{frame}
\begin{block}
  These are all good questions. But remember.
\end{block}

\begin{frame}
  \frametitle{Don't re-implement common tools}
  \begin{itemize}
  \item If you are thinking, ``surely others will have needed this before me'',
    you are most likely right. For advanced functionality \textbf{use existing and commony used implementations}
  \item You are likely you miss a few corner cases that will hurt you later
  \end{itemize}
  \begin{block}{}
    Have a look at $\texttt{joblib}$, which is in pip.
  \end{block}
\end{frame}

\begin{frame}
  \frametitle{And then the mandatory Spiderman quote}
  \begin{block}{}
    With great power comes great responsibility.
  \end{block}
  \begin{itemize}
  \item Don't hide \textbf{large} amounts of code away, that is central to the
    function of your code. It will make it more difficult to read.
  \item Caching is a great way to end up with some head-scratching bugs. Make
    sure to understand the limitations of a tool like $\texttt{joblib}$.
  \item \textbf{But}, if in doubt, just remove the decorator. It is just 1
    line, that toggles the added functionality off or on. Thats one of the
    beauties of decorators
  \end{itemize}
\end{frame}

\begin{frame}
  \frametitle{And then the stuff we did not talk about}
  \begin{block}{}
    A decorator returns another functions, which means that function
    'metadata' like its __name__, its docstring and its call specification is
    lost!
  \end{block}
  There are (ready to use) ways around this (stdlib $\texttt{wraps}$, pip
  $\texttt{wrapt}$), of which at least the former is a part on any good online
  tutorial.
\end{frame}

\end{document}